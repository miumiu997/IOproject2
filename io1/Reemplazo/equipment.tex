\documentclass[10]{beamer} 
\usepackage[T1]{fontenc} 
\usepackage{lmodern}
\usepackage[spanish]{babel}
\usepackage{fancyhdr}
\usepackage{xcolor}
\usepackage{color}
\usepackage{dirtytalk}
\definecolor{dgreen}{rgb}{0.,0.6,0.}
\definecolor{WHITE}{RGB}{255,255,255}
\definecolor{amarillo}{RGB}{255,255,0}
\definecolor{gris}{RGB}{174,174,174}
\definecolor{ROJO}{RGB}{237,28,36}
\setbeamercolor{normal text}{bg=black!80}
\setbeamercolor{frametitle}{fg=black,bg=amarillo!20}
\setbeamercolor{subtitle}{fg=WHITE}
\setbeamercolor{section in head/foot}{bg=amarillo}
\setbeamercolor{author}{fg=WHITE}
\setbeamercolor{date in head/foot}{fg=amarillo}
\title{{\color{WHITE} \large \textbf{INSTITUTO TECNOL\'OGICO DE COSTA RICA}} \\ \vspace{0.02cm} 
{\color{WHITE} \large \textbf{ESCUELA DE INGENIER\'IA EN COMPUTACI'ON }} \\ \vspace{0.02cm} 
{\color{WHITE} \large \textbf{INVESTIGACI\'ON DE OPERACIONES  }} \\ \vspace{0.02cm} 
{\color{WHITE} \large \textbf{ALGORITMO PARA REEMPLAZO \\ DE EQUIPOS  }} \\ \vspace{0.02cm} 
{\color{WHITE} \large \textbf{I SEMESTRE  }}\\ \vspace{0.02cm}
{\color{WHITE} \large \textbf{PROFESOR}} \\ \vspace{0.02cm}
{\color{WHITE} \large DR. FRANCISCO J. TORRES ROJAS  } \\ \vspace{0.02cm}
{\color{WHITE} \large \textbf{GRUPO 40}} \\ \vspace{0.01cm}
{\color{WHITE} \large \textbf{ESTUDIANTES} }} 
\color{WHITE} \author{KATHY ANDRE\'INA BRENES GUERRERO. \\ ADRIAN CUBERO MORA. \\MIUYIN YONG WONG}
\date{\em \color{WHITE} \today}
\begin{document}
\begin{frame}
\color{white}
\titlepage portada
\end{frame} 
\begin{frame}
\color{white}
\frametitle{FUNCIONAMIENTO}
El empleo del an\'alisis de reemplazo de equipo nos muestra 
\\una categor\'ia de decisiones de inversi\'on que implica 
\\considerar el gasto necesario para reemplazar equipo 
\\ desgastado u obsoleto por tecnolog\'ia de punta que permita
\\ mejorar la eficiencia de la producci\'on y elevar el \'indice de 
\\ productividad.
\end{frame} 
\frame{\frametitle{Reemplazo de Equipos}
 \color{white}
\begin{center}
 \begin{tabular}{|l |l |l |l |}
 \hline
 &G(T)&Proximo\\ \hline
0&640&1, 3\\ \hline
1&510&2, 4\\ \hline
2&380&5\\ \hline
3&260&4\\ \hline
4&130&5\\ \hline
\end{tabular}
\end{center}
 }
\frame{\frametitle{Rutas Optimas}
 \color{white}
 1 -2 -5 \newline 1 -4 -5 \newline 3 -4 -5 \newline}
\end{document}
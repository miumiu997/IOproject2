\documentclass[10]{beamer} 
\usepackage[T1]{fontenc} 
\usepackage{lmodern}
\usepackage[spanish]{babel}
\usepackage{fancyhdr}
\usepackage{xcolor}
\usepackage{color}
\usepackage{dirtytalk}
\definecolor{dgreen}{rgb}{0.,0.6,0.}
\definecolor{WHITE}{RGB}{255,255,255}
\definecolor{amarillo}{RGB}{255,255,0}
\definecolor{gris}{RGB}{174,174,174}
\definecolor{ROJO}{RGB}{237,28,36}
\setbeamercolor{normal text}{bg=black!80}
\setbeamercolor{frametitle}{fg=black,bg=amarillo!20}
\setbeamercolor{subtitle}{fg=WHITE}
\setbeamercolor{section in head/foot}{bg=amarillo}
\setbeamercolor{author}{fg=WHITE}
\setbeamercolor{date in head/foot}{fg=amarillo}
\title{{\color{WHITE} \large \textbf{INSTITUTO TECNOL\'OGICO DE COSTA RICA}} \\ \vspace{0.02cm} 
{\color{WHITE} \large \textbf{ESCUELA DE INGENIER\'IA EN COMPUTACI'ON }} \\ \vspace{0.02cm} 
{\color{WHITE} \large \textbf{INVESTIGACI\'ON DE OPERACIONES  }} \\ \vspace{0.02cm} 
{\color{WHITE} \large \textbf{ALGORITMO PARA LA \\ MULTIPLICACI\'ON DE MATRICES  }} \\ \vspace{0.02cm} 
{\color{WHITE} \large \textbf{I SEMESTRE  }}\\ \vspace{0.02cm}
{\color{WHITE} \large \textbf{PROFESOR}} \\ \vspace{0.02cm}
{\color{WHITE} \large DR. FRANCISCO J. TORRES ROJAS  } \\ \vspace{0.02cm}
{\color{WHITE} \large \textbf{GRUPO 40}} \\ \vspace{0.01cm}
{\color{WHITE} \large \textbf{ESTUDIANTES} }} 
\color{WHITE} \author{KATHY ANDRE\'INA BRENES GUERRERO. \\ ADRIAN CUBERO MORA. \\MIUYIN YONG WONG}
\date{\em \color{WHITE} \today}
\begin{document}
\begin{frame}
\color{white}
\titlepage portada
\end{frame} 
\begin{frame}
\color{white}
\frametitle{ALGORITMO PARA LA MULTIPLICACI\'ON DE DOS MATRICES}
El n\'umero de columnas de la primera deben ser iguales al n\'umero de filas de la segunda.
\\ Si las dimensiones de la primera son n * k la segunda debe ser k * m.
\\ La matriz resultante es n * m.
\end{frame} 
\begin{frame}
\color{white}
\frametitle{PROPIEDADES DE LA MULTIPLICACI\'ON DE MATRICES}
Se realizan productos puntos entre vectores fila y vectores columna para encontrar cada entrada de la matriz resultante.
\\ No es conmutativa.
\\Un \'arbol binario no vac\'io, de ra\'iz R.
\\ Total de multiplicaciones requeridas? \\ n*k*m  \\ \t con $M_1:n*k; M_2: k*m$.
\end{frame} 
\begin{frame}
\color{white}
\frametitle{MULTIPLICACI\'ON EN CADENA}
Se pueden multiplicar varias matrices en sucesi\'on.
\\ No es conmutativa, pero si asociativa, es decir, se puede hacer las multiplicaciones en diversos \'ordenes.
\\ Distinto orden implica distinto costo.
\end{frame} 
\begin{frame}\frametitle{VALORES INICIALES}
 \color{white}
\begin{table}
 \begin{tabular}{ c | c  | c  | c }
 \\  $S_1$ & $S_2$   & $S_3$   & $S_4$  \\ 
 \hline \hline 
 $10x100$& $100x5$& $5x50$& $50x1$ \\ 
  $d_0xd_1$& $d_1xd_2$& $d_2xd_3$& $d_3xd_4$ 
 \end{tabular}
 \color{white}
\caption{Valores iniciales}
 \end{table}
 \end{frame} 
\begin{frame}\frametitle{TABLA M}
 \color{white}
\begin{table}
 \begin{tabular}{ c | c  | c  | c  | c }
 \\    & 1   & 2   & 3   & 4  \\ 
 \hline \hline 
  1 & 0& 5000& 7500& 1750 \\ 
  2  & & 0& 25000& 750 \\ 
  3  &  & & 0& 250 \\ 
  4  &  &  & & 0 \\ 
  5  &  &  &  &  \\ 
  
 \end{tabular}
 \color{white}
\caption{Tabla P}
 \end{table}
 \end{frame} 
\begin{frame}\frametitle{TABLA P}
 \color{white}
\begin{table}
 \begin{tabular}{ c | c  | c  | c  | c }
 \\    & 1   & 2   & 3   & 4  \\ 
 \hline \hline 
  1  & & 1& 2& 1 \\ 
  2  &  & & 2& 2 \\ 
  3  &  &  & & 3 \\ 
  4  &  &  &  &  \\ 
  5  &  &  &  &  \\ 
  
 \end{tabular}
 \color{white}
\caption{Tabla P}
 \end{table}
 \end{frame} 
\end{document}
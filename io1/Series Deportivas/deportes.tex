\documentclass[10]{beamer} 
\usepackage[T1]{fontenc} 
\usepackage{lmodern}
\usepackage[spanish]{babel}
\usepackage{fancyhdr}
\usepackage{xcolor}
\usepackage{color}
\usepackage{dirtytalk}
\definecolor{dgreen}{rgb}{0.,0.6,0.}
\definecolor{WHITE}{RGB}{255,255,255}
\definecolor{amarillo}{RGB}{255,255,0}
\definecolor{gris}{RGB}{174,174,174}
\definecolor{ROJO}{RGB}{237,28,36}
\setbeamercolor{normal text}{bg=black!80}
\setbeamercolor{frametitle}{fg=black,bg=amarillo!20}
\setbeamercolor{subtitle}{fg=WHITE}
\setbeamercolor{section in head/foot}{bg=amarillo}
\setbeamercolor{author}{fg=WHITE}
\setbeamercolor{date in head/foot}{fg=amarillo}
\title{{\color{WHITE} \large \textbf{INSTITUTO TECNOL\'OGICO DE COSTA RICA}} \\ \vspace{0.02cm} 
{\color{WHITE} \large \textbf{ESCUELA DE INGENIER\'IA EN COMPUTACI'ON }} \\ \vspace{0.02cm} 
{\color{WHITE} \large \textbf{INVESTIGACI\'ON DE OPERACIONES  }} \\ \vspace{0.02cm} 
{\color{WHITE} \large \textbf{ALGORITMO PARA LAS \\ SERIES DEPORTIVAS  }} \\ \vspace{0.02cm} 
{\color{WHITE} \large \textbf{I SEMESTRE  }}\\ \vspace{0.02cm}
{\color{WHITE} \large \textbf{PROFESOR}} \\ \vspace{0.02cm}
{\color{WHITE} \large DR. FRANCISCO J. TORRES ROJAS  } \\ \vspace{0.02cm}
{\color{WHITE} \large \textbf{GRUPO 40}} \\ \vspace{0.01cm}
{\color{WHITE} \large \textbf{ESTUDIANTES} }} 
\color{WHITE} \author{KATHY ANDRE\'INA BRENES GUERRERO. \\ ADRIAN CUBERO MORA. \\MIUYIN YONG WONG}
\date{\em \color{WHITE} \today}
\begin{document}
\begin{frame}
\color{white}
\titlepage portada
\end{frame} 
\begin{frame}
\color{white}
\frametitle{ALGORITMO PARA SERIE DEPORTIVAS}
Entre dos equipos A y B se dice que A tiene una posibilidad p de ganar y B tiene una posibilidad q = 1-p de ganar.
\\ Cu\'al es la probabilidad de que el grupo A sea campe\'on ?
\\Este algoritmo permite contestar esa pregunta.
\end{frame} 
\begin{frame}
\color{white}
\frametitle{PROPIEDADES DE LAS SERIES DEPORTIVAS}
La cantidad de filas y columnas deben ser igual a la canitdad de partidos comenzando desde 0.
\\ Un par ordenado ser\'ia (i,j) donde i representa la cantidad de partidos que le faltan a A para ganar y j representa la cantiad de partidos que le faltan a B para ganar.
\\El estado inicial ser\'ia (cantidad de partidos, cantidad de partidos).
\\ En cada una de esas casillas se calulara la probabilidad que tiene el equipo A en ganar (cada probabilidad es independiente).
\end{frame} 
\begin{frame}
\color{white}
\frametitle{FUNCIONES}
La funci\'on que regresa la probabilidad de que A sea campi\'on se define como:
\\F(i,j)=p*F(i-1,j) + q*F(i, j-1) 
\\ Casos triviales: F(0,j) = 1    F(i,0) = 0
\end{frame} 
\begin{frame}
\color{white}
\frametitle{VENTAJA LOCAL}
Los equipos tienden a jugar mejor en casa en ciertos deportes.
\\Por esa raz\'on se establecen dos probabilidades Ph y Pr. 
\\ Ph siendo en casa y Pr de visita, las probabilidades del equipo contrario son los complementos.
\\ Dada una serie de cuantos equipos ocurriran en casa, se calculan cu\'al de las dos probabildiades se tiene que usar.
\end{frame} 
\end{document}
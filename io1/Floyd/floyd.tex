\documentclass{beamer}
\usepackage{color}
\begin{document}
\title{Algoritmo Floyd}
 \author{Kathy Brenes, Miuyin, Adrian Cubero}
\date{\today}
 \frame{\titlepage}
 \begin{frame}
\color{black}
\frametitle{ALGORITMO DE LAS RUTAS M\'AS CORTAS}
El algoritmo de Floyd-Warshall, descrito en 1959 por Bernard Roy.
\\Es un algoritmo de análisis sobre grafos para encontrar el camino mínimo en grafos dirigidos ponderados.
\\El algoritmo encuentra el camino entre todos los pares de vértices en una única ejecución.
\\ El algoritmo de Floyd-Warshall es un ejemplo de programación dinámica.
\end{frame} 
\frame{\frametitle{D0}
 \begin{center}
 \begin{tabular}{|l |l |} \hline
0&1\\ \hline
1&0\\ \hline
\end{tabular}
\end{center}
 }
\frame{\frametitle{DN}
 \begin{center}
 \begin{tabular}{|l |l |} \hline
0 &1 \\ \hline
1 &0 \\ \hline
\end{tabular}
\end{center}
 }
\frame{\frametitle{P}
 \begin{center}
 \begin{tabular}{|l |l |} \hline
0&0\\ \hline
0&0\\ \hline
\end{tabular}
\end{center}
 }
\frame{\frametitle{rutas}
 nodo:1 - nodo:1
\newline
nodo:1 - nodo:2
\newline
nodo:2 - nodo:1
\newline
nodo:2 - nodo:2
\newline
}
\frame{\frametitle{D1}
 \begin{center}
 \begin{tabular}{|l |l |} \hline
0&1\\ \hline
1&0\\ \hline
\end{tabular}
\end{center}
 }
\frame{\frametitle{D2}
 \begin{center}
 \begin{tabular}{|l |l |} \hline
0&1\\ \hline
1&0\\ \hline
\end{tabular}
\end{center}
 }
\end{document}
\documentclass[10]{beamer} 
\usepackage[T1]{fontenc} 
\usepackage{lmodern}
\usepackage[spanish]{babel}
\usepackage{fancyhdr}
\usepackage{xcolor}
\usepackage{color}
\usepackage{dirtytalk}
\definecolor{dgreen}{rgb}{0.,0.6,0.}
\definecolor{WHITE}{RGB}{255,255,255}
\definecolor{amarillo}{RGB}{255,255,0}
\definecolor{gris}{RGB}{174,174,174}
\definecolor{ROJO}{RGB}{237,28,36}
\setbeamercolor{normal text}{bg=black!80}
\setbeamercolor{frametitle}{fg=black,bg=amarillo!20}
\setbeamercolor{subtitle}{fg=WHITE}
\setbeamercolor{section in head/foot}{bg=amarillo}
\setbeamercolor{author}{fg=WHITE}
\setbeamercolor{date in head/foot}{fg=amarillo}
\title{{\color{WHITE} \large \textbf{INSTITUTO TECNOL\'OGICO DE COSTA RICA}} \\ \vspace{0.02cm} 
{\color{WHITE} \large \textbf{ESCUELA DE INGENIER\'IA EN COMPUTACI'ON }} \\ \vspace{0.02cm} 
{\color{WHITE} \large \textbf{INVESTIGACI\'ON DE OPERACIONES  }} \\ \vspace{0.02cm} 
{\color{WHITE} \large \textbf{ALGORITMO DE A\'RBOLES BINARIOS DE B\'USQUEDAS \'OPTIMAS  }} \\ \vspace{0.02cm} 
{\color{WHITE} \large \textbf{I SEMESTRE  }}\\ \vspace{0.02cm}
{\color{WHITE} \large \textbf{PROFESOR}} \\ \vspace{0.02cm}
{\color{WHITE} \large DR. FRANCISCO J. TORRES ROJAS  } \\ \vspace{0.02cm}
{\color{WHITE} \large \textbf{GRUPO 40}} \\ \vspace{0.01cm}
{\color{WHITE} \large \textbf{ESTUDIANTES} }} 
\color{WHITE} \author{KATHY ANDRE\'INA BRENES GUERRERO. \\ ADRIAN CUBERO MORA. \\MIUYIN YONG WONG}
\date{\em \color{WHITE} \today}
\begin{document}
\begin{frame}
\color{white}
\titlepage portada
\end{frame} 
\begin{frame}
\color{white}
\frametitle{ALGORITMO DE A\'RBOLES BINARIOS DE B\'USQUEDAS \'OPTIMAS}
Tambi\'en llamados BST (acr\'onimo del ingl\'es Binary Search Tree).
\\ Tipo particular de \'arbol binario que presenta una estructura de datos en forma de \'arbol .
\\Un \'arbol binario no vac\'io, de ra\'iz R.
\\ Existe una relaci\'on de orden establecida entre los elementos de los nodos.
\end{frame} 
\begin{frame}
\color{white}
\frametitle{CONDICIONES DE UN \'ARBOL BINARIO}
En caso de tener sub\'arbol izquierdo, la ra\'iz R debe ser mayor que el valor m\'aximo almacenado en el sub\'arbol izquierdo, y que el sub\'arbol izquierdo sea un \'arbol binario de b\'usqueda.
\end{frame} 
\begin{frame}
\color{white}
\frametitle{CONDICIONES DE UN \'ARBOL BINARIO - CONT}
En caso de tener sub\'arbol derecho, la ra\'iz R debe ser menor que el valor m\'inimo almacenado en el sub\'arbol derecho, y que el sub\'arbol derecho sea un \'arbol binario de b\'usqueda.
\end{frame} 
\begin{frame}
\color{white}
\frametitle{PRINCIPIO DE OPTIMALIDAD}
Todos los sub\'arboles de un \'arbol \'optimo son \'optimos con respecto a las claves que contienen.
\end{frame} 
\begin{frame}\frametitle{VALORES INICIALES}
 \color{white}
\end{frame} 
\end{document}
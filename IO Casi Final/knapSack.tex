\documentclass[10]{beamer} 
\usepackage[T1]{fontenc} 
\usepackage{lmodern}
\usepackage[spanish]{babel}
\usepackage{fancyhdr}
\usepackage{xcolor}
\usepackage{color}
\usepackage{dirtytalk}
\definecolor{dgreen}{rgb}{0.,0.6,0.}
\definecolor{WHITE}{RGB}{255,255,255}
\definecolor{amarillo}{RGB}{255,255,0}
\definecolor{gris}{RGB}{174,174,174}
\definecolor{ROJO}{RGB}{237,28,36}
\setbeamercolor{normal text}{bg=black!80}
\setbeamercolor{frametitle}{fg=black,bg=amarillo!20}
\setbeamercolor{subtitle}{fg=WHITE}
\setbeamercolor{section in head/foot}{bg=amarillo}
\setbeamercolor{author}{fg=WHITE}
\setbeamercolor{date in head/foot}{fg=amarillo}
\title{{\color{WHITE} \large \textbf{INSTITUTO TECNOL\'OGICO DE COSTA RICA}} \\ \vspace{0.02cm} 
{\color{WHITE} \large \textbf{ESCUELA DE INGENIER\'IA EN COMPUTACI'ON }} \\ \vspace{0.02cm} 
{\color{WHITE} \large \textbf{INVESTIGACI\'ON DE OPERACIONES  }} \\ \vspace{0.02cm} 
{\color{WHITE} \large \textbf{ALGORITMO DE LA MOCHILA - KNAPSACK  }} \\ \vspace{0.02cm} 
{\color{WHITE} \large \textbf{I SEMESTRE  }}\\ \vspace{0.02cm}
{\color{WHITE} \large \textbf{PROFESOR}} \\ \vspace{0.02cm}
{\color{WHITE} \large DR. FRANCISCO J. TORRES ROJAS  } \\ \vspace{0.02cm}
{\color{WHITE} \large \textbf{GRUPO 40}} \\ \vspace{0.01cm}
{\color{WHITE} \large \textbf{ESTUDIANTES} }} 
\color{WHITE} \author{KATHY ANDRE\'INA BRENES GUERRERO. \\ ADRIAN CUBERO MORA. \\MIUYIN YONG WONG}
\date{\em \color{WHITE} \today}
\begin{document}
\begin{frame}
\color{white}
\titlepage portada
\end{frame} 
\begin{frame}
\color{white}
\frametitle{ALGORITMO DE LA MOCHILA}
El algoritmo de Floyd-Warshall, descrito en 1959 por Bernard Roy.
\\Es un algoritmo de an\'alisis sobre grafos para encontrar el camino m\'inimo en grafos dirigidos ponderados.
\\El algoritmo encuentra el camino entre todos los pares de v\'ertices en una \'unica ejecuci\'on.
\\ El algoritmo de Floyd-Warshall es un ejemplo de programaci\'on din\'amica.
\end{frame} 
\frame{\frametitle{knapSack}
 \begin{center}
 \begin{tabular}{|l |l |l |l |} \hline
\color{red}0&\color{red}0&\color{red}0\\ \hline
\color{red}0&\color{red}0&\color{red}0\\ \hline
\color{red}0&\color{red}0&\color{red}0\\ \hline
\color{red}0&\color{green}7&\color{red}7\\ \hline
\color{green}11&\color{red}11&\color{red}11\\ \hline
\color{green}11&\color{red}11&\color{green}12\\ \hline
\color{green}11&\color{red}11&\color{green}12\\ \hline
\color{green}11&\color{green}18&\color{red}18\\ \hline
\color{green}11&\color{green}18&\color{green}19\\ \hline
\color{green}11&\color{green}18&\color{green}23\\ \hline
\color{green}11&\color{green}18&\color{green}23\\ \hline
\end{tabular}
\end{center}
 }
\end{document}